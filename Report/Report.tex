%----------------------------------------------------------------------------------------
%	PACKAGES AND OTHER DOCUMENT CONFIGURATIONS
%----------------------------------------------------------------------------------------
\documentclass[11pt,a4paper,english,twoside,twocolumn]{article}
\pagenumbering{arabic}
\usepackage{setspace}
%\onehalfspace
\usepackage[hmarginratio=1:1,top=32mm,columnsep=20pt]{geometry}
\usepackage{fancyhdr}
\usepackage{multirow}
\usepackage{multicol}
\usepackage[section]{placeins}
%--Text and Style------------------------------------------------------------------------
\usepackage{microtype} 			% Slightly tweak font spacing for aesthetics
\usepackage[english]{babel} 		% Language hyphenation and typographical rules
\usepackage[T1]{fontenc}
\usepackage[utf8]{inputenc}
\usepackage{ae}
\usepackage{relsize}
\usepackage{csquotes} 
\usepackage{amsmath}
\usepackage{amsfonts}
\usepackage{mathdots}
\usepackage{mathtools}
\usepackage[colorlinks=true]{hyperref}
\hypersetup{
	bookmarksnumbered=true,
	linkcolor=black,
	citecolor=black,
	%pagecolor=black,
	urlcolor=black,
}
\usepackage{verbatim}
\usepackage{alltt}
\DeclareMathOperator{\sgn}{sgn}
\DeclareMathOperator{\RealNumber}{\rm I\!R}
\DeclarePairedDelimiter{\abs}{\lvert}{\rvert}
\DeclarePairedDelimiter{\norma}{\lVert}{\rVert}

%---Figure------------------------------------------------------------------------------

\usepackage{graphicx}
\graphicspath{{./imgs/}}
\renewcommand{\figurename}{Fig.}
\usepackage{subfig}

%---------------------------------------------------------------------------------------
\usepackage{algorithmicx}
\usepackage[ruled]{algorithm}
\usepackage{algpseudocode}
\usepackage{listings}
\usepackage{xcolor}
\definecolor{maroon}{cmyk}{0, 0.87, 0.68, 0.32}
\definecolor{halfgray}{gray}{0.55}
\definecolor{ipython_frame}{RGB}{207, 207, 207}
\definecolor{ipython_bg}{RGB}{247, 247, 247}
\definecolor{ipython_red}{RGB}{186, 33, 33}
\definecolor{ipython_green}{RGB}{0, 128, 0}
\definecolor{ipython_cyan}{RGB}{64, 128, 128}
\definecolor{ipython_purple}{RGB}{170, 34, 255}
\lstset{
    breaklines=true,
    %
    extendedchars=true,
    literate=
    {á}{{\'a}}1 {é}{{\'e}}1 {í}{{\'i}}1 {ó}{{\'o}}1 {ú}{{\'u}}1
    {Á}{{\'A}}1 {É}{{\'E}}1 {Í}{{\'I}}1 {Ó}{{\'O}}1 {Ú}{{\'U}}1
    {à}{{\`a}}1 {è}{{\`e}}1 {ì}{{\`i}}1 {ò}{{\`o}}1 {ù}{{\`u}}1
    {À}{{\`A}}1 {È}{{\'E}}1 {Ì}{{\`I}}1 {Ò}{{\`O}}1 {Ù}{{\`U}}1
    {ä}{{\"a}}1 {ë}{{\"e}}1 {ï}{{\"i}}1 {ö}{{\"o}}1 {ü}{{\"u}}1
    {Ä}{{\"A}}1 {Ë}{{\"E}}1 {Ï}{{\"I}}1 {Ö}{{\"O}}1 {Ü}{{\"U}}1
    {â}{{\^a}}1 {ê}{{\^e}}1 {î}{{\^i}}1 {ô}{{\^o}}1 {û}{{\^u}}1
    {Â}{{\^A}}1 {Ê}{{\^E}}1 {Î}{{\^I}}1 {Ô}{{\^O}}1 {Û}{{\^U}}1
    {œ}{{\oe}}1 {Œ}{{\OE}}1 {æ}{{\ae}}1 {Æ}{{\AE}}1 {ß}{{\ss}}1
    {ç}{{\c c}}1 {Ç}{{\c C}}1 {ø}{{\o}}1 {å}{{\r a}}1 {Å}{{\r A}}1
    {€}{{\EUR}}1 {£}{{\pounds}}1
}

%%
%% Python definition (c) 1998 Michael Weber
%% Additional definitions (2013) Alexis Dimitriadis
%% modified by me (should not have empty lines)
%%
\lstdefinelanguage{iPython}{
    morekeywords={access,and,break,class,continue,def,del,elif,else,except,exec,finally,for,from,global,if,import,in,is,lambda,not,or,pass,print,raise,return,try,while},%
    %
    % Built-ins
    morekeywords=[2]{abs,all,any,basestring,bin,bool,bytearray,callable,chr,classmethod,cmp,compile,complex,delattr,dict,dir,divmod,enumerate,eval,execfile,file,filter,float,format,frozenset,getattr,globals,hasattr,hash,help,hex,id,input,int,isinstance,issubclass,iter,len,list,locals,long,map,max,memoryview,min,next,object,oct,open,ord,pow,property,range,raw_input,reduce,reload,repr,reversed,round,set,setattr,slice,sorted,staticmethod,str,sum,super,tuple,type,unichr,unicode,vars,xrange,zip,apply,buffer,coerce,intern},%
    %
    sensitive=true,%
    morecomment=[l]\#,%
    morestring=[b]',%
    morestring=[b]",%
    %
    morestring=[s]{'''}{'''},% used for documentation text (mulitiline strings)
    morestring=[s]{"""}{"""},% added by Philipp Matthias Hahn
    %
    morestring=[s]{r'}{'},% `raw' strings
    morestring=[s]{r"}{"},%
    morestring=[s]{r'''}{'''},%
    morestring=[s]{r"""}{"""},%
    morestring=[s]{u'}{'},% unicode strings
    morestring=[s]{u"}{"},%
    morestring=[s]{u'''}{'''},%
    morestring=[s]{u"""}{"""},%
    %
    % {replace}{replacement}{lenght of replace}
    % *{-}{-}{1} will not replace in comments and so on
    literate=
    {á}{{\'a}}1 {é}{{\'e}}1 {í}{{\'i}}1 {ó}{{\'o}}1 {ú}{{\'u}}1
    {Á}{{\'A}}1 {É}{{\'E}}1 {Í}{{\'I}}1 {Ó}{{\'O}}1 {Ú}{{\'U}}1
    {à}{{\`a}}1 {è}{{\`e}}1 {ì}{{\`i}}1 {ò}{{\`o}}1 {ù}{{\`u}}1
    {À}{{\`A}}1 {È}{{\'E}}1 {Ì}{{\`I}}1 {Ò}{{\`O}}1 {Ù}{{\`U}}1
    {ä}{{\"a}}1 {ë}{{\"e}}1 {ï}{{\"i}}1 {ö}{{\"o}}1 {ü}{{\"u}}1
    {Ä}{{\"A}}1 {Ë}{{\"E}}1 {Ï}{{\"I}}1 {Ö}{{\"O}}1 {Ü}{{\"U}}1
    {â}{{\^a}}1 {ê}{{\^e}}1 {î}{{\^i}}1 {ô}{{\^o}}1 {û}{{\^u}}1
    {Â}{{\^A}}1 {Ê}{{\^E}}1 {Î}{{\^I}}1 {Ô}{{\^O}}1 {Û}{{\^U}}1
    {œ}{{\oe}}1 {Œ}{{\OE}}1 {æ}{{\ae}}1 {Æ}{{\AE}}1 {ß}{{\ss}}1
    {ç}{{\c c}}1 {Ç}{{\c C}}1 {ø}{{\o}}1 {å}{{\r a}}1 {Å}{{\r A}}1
    {€}{{\EUR}}1 {£}{{\pounds}}1
    %
    {^}{{{\color{ipython_purple}\^{}}}}1
    {=}{{{\color{ipython_purple}=}}}1
    %
    {+}{{{\color{ipython_purple}+}}}1
    {*}{{{\color{ipython_purple}$^\ast$}}}1
    {/}{{{\color{ipython_purple}/}}}1
    %
    {+=}{{{+=}}}1
    {-=}{{{-=}}}1
    {*=}{{{$^\ast$=}}}1
    {/=}{{{/=}}}1,
    literate=
    *{-}{{{\color{ipython_purple}-}}}1
     {?}{{{\color{ipython_purple}?}}}1,
    %
    identifierstyle=\color{black}\ttfamily,
    commentstyle=\color{ipython_cyan}\ttfamily,
    stringstyle=\color{ipython_red}\ttfamily,
    keepspaces=true,
    showspaces=false,
    showstringspaces=false,
    %
    rulecolor=\color{ipython_frame},
    frame=single,
    frameround={t}{t}{t}{t},
    framexleftmargin=6mm,
    numbers=left,
    numberstyle=\tiny\color{halfgray},
    %
    %
    backgroundcolor=\color{ipython_bg},
    %   extendedchars=true,
    basicstyle=\scriptsize,
    keywordstyle=\color{ipython_green}\ttfamily,
}

%------Matlab scheme color-code----------------------------------
\usepackage{matlab-prettifier}
\lstset{
	style = Matlab-editor,
	basicstyle=\mlttfamily,
    %Style frame and number
    rulecolor = \color{ipython_frame},
    frame	=single,
    frameround={t}{t}{t}{t},
    framexleftmargin=6mm,
    numbers=left,
    numberstyle=\tiny\color{halfgray},
	%background color frame
    backgroundcolor=\color{ipython_bg}   
}

%---Table------------------------------------------------------------
\usepackage{tabularx}
\usepackage{array}
\usepackage{color}
\usepackage{adjustbox}
\usepackage{colortbl}
\usepackage{pgfplotstable}
\usepackage{makecell}
\usepackage{booktabs}

%--------------------------------------------------------------------
% Allows abstract customization
\usepackage{abstract}

% Set the "Abstract" text to bold
\renewcommand{\abstractnamefont}{\normalfont\bfseries} 
% Set the abstract itself to small italic text
\renewcommand{\abstracttextfont}{\normalfont\small\itshape} 

% Allows customization of titles
\usepackage{titlesec} 

% Roman numerals for the sections
\renewcommand\thesection{\Roman{section}} 

% roman numerals for subsections
\renewcommand\thesubsection{\roman{subsection}} 

% Change the look of the section titles
\titleformat{\section}[block]{\large\scshape\centering}{\thesection.}{1em}{} 

% Change the look of the section titles
\titleformat{\subsection}[block]{\large}{\thesubsection.}{1em}{} 
\usepackage{titling} % Customizing the title section

%----------------------------------------------------------------------------------------
\usepackage{tikz,fp,ifthen,fullpage}
\usepackage{pgfmath, pgfplots, xparse}
\usepgfplotslibrary{fillbetween}
\usetikzlibrary{backgrounds, arrows}
\usetikzlibrary{decorations.pathmorphing,fit,through}
\usetikzlibrary{shapes,decorations,shadows}
\usetikzlibrary{fadings,patterns,mindmap}
\usepackage{tikz-dimline, calc}
\pgfplotsset{compat=newest}

%----------------------------------------------------------------------------------------
%	TITLE SECTION
%----------------------------------------------------------------------------------------
\setlength{\droptitle}{-4\baselineskip} % Move the title up


\pretitle{\begin{center}\Huge\bfseries} % Article title formatting


\posttitle{\end{center}} % Article title closing formatting


\title{Homework - Kinematics Calibration of a Differential drive wheelchair} % Article title

\author{%
	% Your name
	\textsc{Francesco Argentieri}\thanks{ID: 183892}\\[1ex] 
	% Your institution
	\normalsize University of Trento \\ 
	% Your email address
	\normalsize \href{mailto:francesco.argentieri@studenti.unitn.it}{francesco.argentieri@studenti.unitn.it} 
}
%\and % Uncomment if 2 authors are required, duplicate these 4 lines if more
%\textsc{Jane Smith}\thanks{Corresponding author} \\[1ex] % Second author's name
%\normalsize University of Utah \\ % Second author's institution
%\normalsize \href{mailto:jane@smith.com}{jane@smith.com} % Second author's email address

% Leave empty to omit a date
\date{} 
\renewcommand{\maketitlehookd}{%
	\begin{abstract}
		\noindent  For a mobile robot, odometry calibration consists of the identification of a set of kinematic parameters that allow reconstructing the vehicle’s relative position and orientation starting from the wheels’ encoder measurements.
The aim of the homework is to:
estimate from the encoder and the camera data the kinematics parameters wheels radius “$r_\textsc{l}$" and “$r_\textsc{r}$" and wheelbase “b".
Estimate the camera position with respect to the wheelchair reference point (the mid point between the wheels).
\end{abstract}
}

%----------------------------------------------------------------------------------------
\begin{document}
	% Print the title
	\maketitle
	
	%Other chapter
	\section{Introduction}
The document presented is based on the analysis of a real robot, developed and built at the MIRO - Measurements Instrumentations Robotics Lab at the University of Trento.
	\section{Methods}
\subsection{Kinematic Model}
For the analysis of differential wheelchair drive parameters using a simplified model in the reference system prorprio odometric placed in the center of the robot indicated by $RF1$. Consider that moves in a reference system fixed to the $RF0$ envoiroment.
The robot is subject to pure rolling, then we neglect slippage between wheel and ground. The angular velocities, indicated respectively with omega, are applied respectively to the right and left wheels in such a way that the components of the fixed body and the speed of the robot are related to the angular velocity of the wheels according to the equation functional (\ref{eq:velocityrelation}).
The other two support wheels are considered passive. This schematization is observable in the fig. \ref{fig:model}.
\begin{equation}
\label{eq:velocityrelation}
	\left[ \begin{array}{cc}
				v	\\									
				\omega 							
			 \end{array} 
	\right]  =\, C\,
	\left[ \begin{array}{cc}
 				\omega_\textsc{l} \\ 
				\omega_\textsc{r}
			 \end{array} 
	\right]
\end{equation}

\begin{figure}[!h]
\centering
    \resizebox{.8\linewidth}{!}{\begin{tikzpicture} [>=latex]
% \draw [help lines] (0,0) grid (8, 8);
% \foreach \x in {0,1,...,8}
%   \draw [help lines] (\x,0) node [below,%
%          font=\footnotesize] {$\x$} -- (\x,0);
%\foreach \y in {0,1,...,8}
%   \draw [help lines] (0,\y) node [left,%
%          font=\footnotesize] {$\y$} -- (0,\y);
 % body robot
 \draw [fill=lightgray, fill opacity=0.5](4, 4) circle (2.25);
 \draw [rounded corners=15,pattern=crosshatch, pattern color=gray] (0.5, 2.5)  rectangle (1.5, 5.5);
 \draw [rounded corners=15,pattern=crosshatch, pattern color=gray] (6.5, 2.5) rectangle (7.5, 5.5);
 \draw (1.5, 4) -- (6.5, 4);
 % quote wheel
 \dimline  [color=gray, 
                 %line style={thick},
                %extension start style={gray,thin},
                %extension end style={gray,thin},
               extension start length=1cm,
              extension end length=1cm,
                ]{(0, 4)}{ (0, 5.5)}{$r$};
 % quote track
 \dimline    [color=gray,
                % line style={thick},
                %extension start style={gray,thin},
                %extension end style={gray,thin},
                extension start length=-1cm,
                extension end length=-1cm
                ]{(1.5, 1.25)}{ (6.5, 1.25)}{$b$};
 % vector
 	\draw [->, blue] (4, 6) -- (4, 8) node[left]{$v$};
 	\draw [->, red] (1, 4) -- (1,7) node[left]{$\omega_{r}$};
 	\draw [->, red] (7, 4) -- (7,7)	 node[left]{$\omega_{l}$};
	\draw [->, orange] (4.75,4) arc (0:(-300+45):0.75) node[below]{$\omega$};
 % Reference system 0
 	\coordinate [label = below: \scriptsize $RF0$] (A) at (0,0);
 	\coordinate	(Bx) at	($(A)+1*(0:1)$);
 	\coordinate	(By)	at	($(A)+1*(90:1)$);
 	\draw [->] 	(A) -- (By) node[left]{$y$};
 	\draw [->] 	(A) -- (Bx) node[above]{$x$};
 % RF1
  	\coordinate [label = below: \scriptsize $RF1$] (RF1) at (4,4);
 	\coordinate	(RF1x) at	($(RF1)+1*(0:1)$);
 	\coordinate	(RF1y)	at	($(RF1)+1*(90:1)$);
 	\fill(RF1) circle (1.4pt);
 	\draw [->] (RF1) -- (RF1y) node[left] {$y'$};
 	\draw [->] (RF1) -- (RF1x) node[above] {$x'$};
 % RFcam
 	\coordinate [label = below: \scriptsize $RF_{cam}$] (R) at (4.3,2.8);
 	\coordinate (Rx) at ($(R)+1*(22.5:1)$);
 	\coordinate (Ry) at ($(R)+1*(90+22.5:1)$);
 	\fill (R) circle (1.4pt);
	 \draw [->, darkgray] (R) -- (Ry) node[left] {$y_{cam}$};
 	 \draw [->, darkgray] (R) -- (Rx) node[above]{$x_{cam}$};
 \end{tikzpicture}}
\caption{Robot kinematic model}
\label{fig:model}
\end{figure}
\noindent Where the matrix $\textbf{C} \in \RealNumber^{2 x 2}$ is defined as (\ref{eq:Cmatrix}):
\begin{equation}
\label{eq:Cmatrix}
	\left[ \begin{array}{cc}
 				\frac{r_\textsc{r}}{2} &	\frac{r_\textsc{l}}{2} \\
				\frac{r_\textsc{l}}{b} &	-\frac{r_\textsc{l}}{b} 
			 \end{array} 
	\right]
\end{equation}
in which $r_\textsc{r}$ and $r_\textsc{l}$ are the radii of the right and left wheel, respectively.
The odometry of a vehicle is usually implemented by discrete-time integration, such as (\ref{eq:odometry}):
\begin{equation}
\label{eq:odometry}
	\begin{cases}
		x_{k+1} = x_{k} + T \, v_{k} \, cos(\theta_{k} + T \,\omega_{k}/2)\\
		y_{k+1} = y_{k} + T \, v_{k} \, sin(\theta_{k} + T \, \omega_{k}/2)\\
		\theta_{k+1} = \theta_{k} + T \, \omega_{k}
	\end{cases}
\end{equation}           
Notice that low sampling frequency and high vehicle velocities can be a significant source of odometric error.
\subsection{Data Analysis}
To achieve the objective of calibration of the geometric parameters of the robot four datasets were provided in each of which the information is collected from the camera and from the incremental encoder.
A first data operation was carried out by a python script to correct the registered angle sign as evaluated with a negative sign.
A second operation was to change the left encoder incremental measuring, since being mounted on reverse respect to the right its value must be taken with a negative sign.
In the file is observed the possibility of the presence of two successive rows with the same timestamp. 
This means that at the same time both the encoder ticks and the camera pose or covariance changed.
It may be noted that in one line the camera information and the other the odometer ticks have changed. 
Thus, at the same timestamp, in the second row the information is modified.
So you chose to eliminate duplicate rows to preserve the last where both information camera pose or covariance and ticks has changed.
\subsection{Calibration Techinque}
To estimate the parameters of the robot expressed in the equation \ref{eq:Cmatrix}, namely, the wheel radius values indicated with "$r_\textsc{r}$" and "$r_\textsc{l}$", and the axle track as indicated "$b$", using the method described in the report \cite{1512356}.
Experiments of odometry calibration require measurement of the absolute position and orientation of the mobile robot at suitable locations along the motion trajectories. For instance this calibration technique requires measurement of the starting and final robot configuration for each motion execution. The fiures \ref{fig:data} shows the data sets supplied information related to: the camera position $x, y, \psi$ and increments of the encoder.
It has been chosen to perform the odometry calibration whereas in the same calculation all four datasets provided as suggested by the technique used. Then the equations \ref{eq:Phi4} and \ref{eq:XY4} are rewritten limited to the four paths to obtain:
\begin{equation}
\label{eq:Phi4}
	\begin{bmatrix}
		\hat{c}_{2,1}\\
		\hat{c}_{2,2}
	\end{bmatrix} =	(\bar{\Phi}_{\theta}^{\textsc{t}} \, \bar{\Phi}_{\theta})^{-1} \, \bar{\Phi}_{\theta}^{\textsc{t}} \, 
	\begin{bmatrix}
		\theta_{\textsc{n},1} - \theta_{\textsc{n},0}\\
		\theta_{\textsc{n},2} - \theta_{\textsc{n},0}\\
		\theta_{\textsc{n},3} - \theta_{\textsc{n},0}\\
		\theta_{\textsc{n},4} - \theta_{\textsc{n},0}
	\end{bmatrix}
\end{equation}
\begin{equation}
\label{eq:XY4}
	\begin{bmatrix}
		\hat{c}_{1,1}\\
		\hat{c}_{1,2}
	\end{bmatrix} = (\bar{\Phi}_{xy}^{\textsc{t}} \, \bar{\Phi}_{xy})^{-1} \, \bar{\Phi}_{xy}^{\textsc{t}} \, 
	\begin{bmatrix}
		xy_{\textsc{n},1} - xy_{\textsc{n},0}\\
		xy_{\textsc{n},2} - xy_{\textsc{n},0}\\
		xy_{\textsc{n},3} - xy_{\textsc{n},0}\\
		xy_{\textsc{n},4} - xy_{\textsc{n},0}
	\end{bmatrix}
\end{equation}
 As a result of simulations shows the values of the matrix C:
\begin{equation}
\label{eq:Cresult}
	\begin{bmatrix}
		8.1873  &  6.5229\\
    	0.2823 &  -0.2807
	\end{bmatrix}
\end{equation}
It is noted that the parameters $c_ {1,1}$ and $c_ {1,2}$ relative to the spokes of the wheels are different from each other, this is due to the simplifications introduced by the kinematic model. On the other hands, the axle values $c_ {2,1}$ and $c_ {2,2}$, without the negative sign, are much more similar because precedenemente estimated at $c_{1,j}$ and therefore do not contain the error propagation.
Subsequently, it calculates the average between the values of the obtained rays and the standard deviation, these are shown in the table (\ref{tab:recapvalue}).
\begin{table}[!h]
\centering
	\begin{tabular}{lccc}
		\hline
								& Radius 	& Mean 	& standard  \\
								&	[mm]	& [mm]	& deviation [mm]\\
		\hline
		$r_\textsc{r}$	&	$130.457$		& $147.102	$		&	$\pm23.538$\\
		$r_\textsc{l}$	&	$163.746$		& $147.102	$		&	$\pm23.538$\\
		$b$					&	$522.43$\\
		\hline
\end{tabular}
\caption{estimated value}
\label{tab:recapvalue}
\end{table}

%
%\begin{figure*}[htb]
%\subfloat[][\emph{dataset 1}.]
%   {\includegraphics[width=.30\textwidth]{angle_dataset_1.eps}} \,
%\subfloat[][\emph{dataset 2}.]
%   {\includegraphics[width=.30\textwidth]{angle_dataset_2.eps}} \,
%\subfloat[][\emph{dataset 3}.]
%   {\includegraphics[width=.30\textwidth]{angle_dataset_3.eps}} \,
%\subfloat[][\emph{dataset 4}.]
%   {\includegraphics[width=.35\textwidth]{angle_dataset_4.eps}}\\
%%\caption{Angle captured by the camera}
%\end{figure*}
%
%\begin{figure*}[htb]
%\subfloat[][\emph{dataset 1}.]
%   {\includegraphics[width=.35\textwidth]{trajectory_dataset_1.eps}} \,
%\subfloat[][\emph{dataset 2}.]
%   {\includegraphics[width=.35\textwidth]{trajectory_dataset_2.eps}} \,
%\subfloat[][\emph{dataset 3}.]
%   {\includegraphics[width=.35\textwidth]{trajectory_dataset_3.eps}} \,
%\subfloat[][\emph{dataset 4}.]
%   {\includegraphics[width=.35\textwidth]{trajectory_dataset_4.eps}} \\
%\end{figure*}
%\begin{figure*}[htb]
%\subfloat[][\emph{dataset 1}.]
%   {\includegraphics[width=.35\textwidth]{tick_dataset_1.eps}} \,
%\subfloat[][\emph{dataset 2}.]
%   {\includegraphics[width=.35\textwidth]{tick_dataset_2.eps}} \,
%\subfloat[][\emph{dataset 3}.]
%   {\includegraphics[width=.35\textwidth]{tick_dataset_3.eps}} \,
%\subfloat[][\emph{dataset 4}.]
%   {\includegraphics[width=.35\textwidth]{tick_dataset_4.eps}}
%\caption{Dataset}
%\label{fig:data}
%\end{figure*}
\noindent
Finally, we show in the figures \ref{fig:OdoRec}, it shows the calculation for each path odometric with parameters previously estimated in comparison with the trajectory recorded by the camera.
\begin{figure*}[htb]
\subfloat[][\emph{dataset 1}.]
   {\includegraphics[width=0.5\textwidth]{odocam_1.eps}} \,
\subfloat[][\emph{dataset 2}.]
   {\includegraphics[width=0.5\textwidth]{odocam_2.eps}} \\
\subfloat[][\emph{dataset 3}.]
   {\includegraphics[width=0.5\textwidth]{odocam_3.eps}} \,
\subfloat[][\emph{dataset 4}.]
   {\includegraphics[width=0.5\textwidth]{odocam_4.eps}}
\caption{Odometry reconstruction}
\label{fig:OdoRec}
\end{figure*}

	\section{Results}
Using the method proposed in scientific article \cite{1512356} allows to obtain the values in the table \ref{tab:recapvalue}; it is observed that the distance between the robot’s actual parameters and the estimates are introduced by the assumed model simplifications and errors.
In the first analysis, the odometric center is placed in mid’s robot axle as well as in the model used, but this method does not take into account that the camera’s position may be not aligned with the point previously analyzed.
Errors also depend on assumptions regarding the non-deformability of the wheels, the tire above keyed and misalignments.\\ 
The motion is achieved by considering a regular and uniform surface, ignoring bumps, obstacles, unevenness, etc. 
In this scientific article, as in other scientific works \cite{censi13joint}--\cite{Jung2016}--\cite{DBLP:journals/ijrr/ChongK99}, is recommanded to use predetermined paths, possibly in a straight line, with counterclockwise and clockwise rotations in order to minimize errors and to allow independent calibration of the parameters.
On the other hand for the results obtained from optimization we must take into account that there are limitations of the use of a genetic algorithm compared to alternative optimization algorithms.
Repeated fitness function evaluation for complex problems is often the most prohibitive and limiting segment of artificial evolutionary algorithms. Finding the optimal solution to complex high-dimensional, multimodal problems often requires very expensive fitness function evaluations. In real world problems such as structural optimisation problems, a single function evaluation may require several hours to several days of complete simulation. Typical optimisation methods can not deal with such types of problem. In this case, it may be necessary to forgo an exact evaluation and use an approximated fitness that is computationally efficient. It is apparent that amalgamation of approximate models may be one of the most promising approaches to convincingly use GA to solve complex real-life problems.
The ``better" solution is only in comparison to other solutions. As a result, the stop criterion is not clear in every problem.
In many problems, GAs may have a tendency to converge towards local optima or even arbitrary points rather than the global optimum of the problem. This means that it does not ``know how" to sacrifice short-term fitness to gain longer-term fitness. The likelihood of this occurring depends on the shape of the fitness landscape: certain problems may provide an easy ascent towards a global optimum, others may make it easier for the function to find the local optima. This problem may be alleviated by using a different fitness function, increasing the rate of mutation, or by using selection techniques that maintain a diverse population of solutions, although the ``\emph{No Free Lunch theorem}" proves that there is no general solution to this problem. A common technique to maintain diversity is to impose a ``niche penalty", wherein, any group of individuals of sufficient similarity (niche radius) have a penalty added, which will reduce the representation of that group in subsequent generations, permitting other (less similar) individuals to be maintained in the population. This trick, however, may not be effective, depending on the landscape of the problem. Another possible technique would be to simply replace part of the population with randomly generated individuals, when most of the population is too similar to each other. Diversity is important in genetic algorithms (and genetic programming) because crossing over a homogeneous population does not yield new solutions. In evolution strategies and evolutionary programming, diversity is not essential because of a greater reliance on mutation.

%----------------------------------------------------------------------------------------
%	Bibliography
%----------------------------------------------------------------------------------------
	
	\bibliographystyle{IEEEtran}
	\bibliography{bibliografy.bib}

%----------------------------------------------------------------------------------------
%	Programming Languages
%---------------------------------------------------------------------------------------- 
	\onecolumn
	\newpage
	\lstinputlisting[language=iPython,label=lang:codePython,caption=main.py]{/Users/francescoargentieri/Kinematics-Calibration-of-a-Differential-drive/Data/PythonScript/main.py}
	\lstinputlisting[language=iPython,label=lang:codePythonPd,caption=dataframe.py]{/Users/francescoargentieri/Kinematics-Calibration-of-a-Differential-drive/Data/PythonScript/dataframe.py}
	\newpage	
	\lstinputlisting[label=lang:MatlabObjFunc,caption=gaerror.m]{/Users/francescoargentieri/Kinematics-Calibration-of-a-Differential-drive/Optimization/gaerror.m}
\end{document}