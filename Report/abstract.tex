\noindent In the development of autonomous driving robot intended for the street or for closed environments where the precision of movement and `extremely important, an element to keep in consideration` and the consciousness that the robot has of itself, ie of its characteristic dimensions, in particular those that affect its kinematic motion. In these pages you will take into consideration some methods that allow the robot itself to be traced back to its size characteristics starting from the location data and motion from various sensors.
\noindent
The odometry calibration consists in the identification of a set of kinematic parameters that allow to reconstruct the absolute position and orientation of the vehicle, based on the measurement and a written confirmation either via the encoder. In this report we show the results obtained by following this approach for the identification of the parameters of a pram.


\noindent For a mobile robot, odometry calibration consists of the identification of a set of kinematic parameters that allow reconstructing the vehicle’s absolute position and orientation starting from the wheels’ encoder measurements. This paper apply a method  for odometry calibration of differential drive mobile robots. As a first step, the kinematic equations are written so as to underline linearity in a suitable set of unknown parameters; thus, the least-squares method can be applied to estimate them. A major advantage of the adopted formulation is that it provides a quantitative measure of the optimality of a test motion. The proposed technique has been experimentally validated on two different mobile robots and, in one case, compared with other existing approaches; the obtained results confirm the effectiveness of the proposed calibration method.
