\section{Results}
Using the scientific article by proposed method allows to obtain the values obtained in the table, it is observed that the distance between the robot's actual parameters and those estimates are introduced by the assumed model simplifications and errors.
In the first analysis the method proposed odometric identifies the center of the robot in dell'axle track centerline as well as the use of the model, but does not take account of the camera with respect to decentralization odometric center.
Errors also depend on assumptions regarding the non-deformability of the wheels and the tire above keyed, any dissallineamenti. The motion is achieved by considering a regular and uniform surface, and then ignoring bumps, obstacles, unevenness, etc.
Also recommend to use predetermined paths, possibly, whereas those in a straight line, with counterclockwise and clockwise rotations in order to minimize errors and to allow independent calibration of the parameters.